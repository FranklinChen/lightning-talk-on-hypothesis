\usepackage{minted}

\title{5-minute intro to property-based testing in Python with \href{https://github.com/DRMacIver/hypothesis}{\texttt{hypothesis}}}
\author{Franklin Chen \\ \url{http://FranklinChen.com/} \\ \url{http://ConscientiousProgrammer.com/}}
\date{\href{http://www.meetup.com/pghpython/events/120442102}{Pittsburgh Python Meetup} \\
June 26, 2013
}

\begin{document}

\begin{frame}
  \titlepage
\end{frame}

% TODO
%\section*{Outline}
%\begin{frame}
%  \frametitle{Outline}
%  \tableofcontents[pausesections]
%\end{frame}

\begin{frame}[fragile]
  \frametitle{Property-based testing vs. example-based testing}
  
  Example-based testing: assert that a property is true for \emph{some} values.
  \begin{itemize}  
  \item Are list appends associative?
  \item Write some \emph{examples}
  \end{itemize}

  \begin{minted}{python}
def test_list_append_associative_example1():
    x, y, z = [2], [3, 4, 5], [6, 7]
    assert (x + y) + z == x + (y + z)

def test_list_append_associative_example2():
    x, y, z = [2, 3], [], [6, 7]
    assert (x + y) + z == x + (y + z)

# ...
  \end{minted}    
\end{frame}

\begin{frame}[fragile]
  \frametitle{Better: reduce boilerplate with parameters}

  \href{http://pytest.org/}{pytest} allows parameterization of arguments in tests.
  \begin{itemize}
  \item Rewrite multiple examples as a \emph{single example}
  \end{itemize}

  \begin{minted}{python}
import pytest

@pytest.mark.parametrize (("x", "y", "z"), [
    ([2], [3, 4, 5], [6, 7]),
    ([2, 3], [], [6, 7])
])
def test_list_append_associative_parametrized(x, y, z):
    assert (x + y) + z == x + (y + z)
  \end{minted}
\end{frame}

\begin{frame}[fragile]
  \frametitle{Introduction to property-based hypothesis testing}

  \begin{minted}{python}
from hypothesis.testdecorators import given

@given ([int], [int], [int])
def test_list_append_associative(x, y, z):
    assert (x + y) + z == x + (y + z)
  \end{minted}

  \begin{itemize}
  \item Passes!
  \item \texttt{hypothesis} \emph{generates} a ``large'' number of examples
  \item Generation
    \begin{itemize}
    \item Based on the \emph{type} of the argument
    \item Not exhaustive: failure to \emph{falsify} does not mean true!
    \item \href{https://github.com/DRMacIver/hypothesis/blob/master/hypothesis/searchstrategy.py}{Default generators provided}
    \item You can write your own generators
    \end{itemize}
  \end{itemize}
\end{frame}

\begin{frame}[fragile]
  \frametitle{A sample falsified hypothesis}

  \begin{minted}{python}
@given (int, int)
def test_multiply_then_divide_is_same(x, y):
    assert (x * y) / y == x
  \end{minted}

  Result:
  \begin{minted}{console}
... falsifying_example = ((0, 0), {})
  \end{minted}
\end{frame}

\begin{frame}
  \frametitle{Some links}

  \begin{itemize}
  \item \texttt{hypothesis} \href{https://pypi.python.org/pypi/hypothesis}{documentation}
  \item \href{https://pypi.python.org/pypi/pytest-quickcheck/}{pytest-quickcheck}
  \item \href{http://franklinchen.com/blog/2013/04/11/my-pittsburgh-scala-meetup-talk-on-property-based-testing-using-scalacheck/}{My Pittsburgh Scala meetup talk on property-based testing using \texttt{ScalaCheck}}
  \item \href{http://natpryce.com/articles/000801.html}{Nat Pryce's blog post on June 23, 2013}
  \end{itemize}
\end{frame}

\begin{frame}
  \frametitle{Conclusion}

  \begin{itemize}  
  \item Property-based testing is a useful addition to your toolbox
  \item Try it out in Python!
  \item All materials for this talk available at \url{https://github.com/franklinchen/lightning-talk-on-hypothesis}
  \item Will write more about property-based testing on new blog \url{http://ConscientiousProgrammer.com/}
  \end{itemize}

\end{frame}

\end{document}
